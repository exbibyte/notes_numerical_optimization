\documentclass[12pt,letter]{article}

%% \usepackage[fleqn]{amsmath}
\usepackage[margin=1in]{geometry}
\usepackage{amsmath,amsfonts,amsthm,bm}
\usepackage{breqn}
\usepackage{amsmath}
\usepackage{amssymb}
\usepackage{tikz}
\usepackage{algorithm2e}
\usepackage{siunitx}
\usepackage{graphicx}
\usepackage{subcaption}
%% \usepackage{datetime}
\usepackage{multirow}
\usepackage{multicol}
\usepackage{mathrsfs}
\usepackage{fancyhdr}
\usepackage{fancyvrb}
\usepackage{parskip} %turns off paragraph indent
\pagestyle{fancy}

\usetikzlibrary{arrows}

\DeclareMathOperator*{\argmin}{argmin}
\newcommand*{\argminl}{\argmin\limits}

\newcommand{\mathleft}{\@fleqntrue\@mathmargin0pt}
\newcommand{\R}{\mathbb{R}}
\newcommand{\Z}{\mathbb{Z}} 
\newcommand{\N}{\mathbb{N}}
\newcommand{\ppartial}[2]{\frac{\partial #1}{\partial #2}}
\newcommand{\p}{\partial}
\newcommand{\te}[1]{\text{#1 }}
\newcommand{\norm}[1]{\|#1\|}

\setcounter{MaxMatrixCols}{20}

% remove excess vertical space for align equations
\setlength{\abovedisplayskip}{0pt}
\setlength{\belowdisplayskip}{0pt}
\setlength{\abovedisplayshortskip}{0pt}
\setlength{\belowdisplayshortskip}{0pt}

\newtheorem{theorem}{Theorem}[section]
\newtheorem{corollary}{Corollary}[theorem]
\newtheorem{lemma}[theorem]{Lemma}

\begin {document}

\lhead{Notes - Numerical Optimization, 2020/01/20}

\section{Taylor series with remainder}
  \[\phi(t), \phi^{(1)}(t), \phi^{(n)}(t), \phi_i \in \R\to\R \]
  exist and are continous for $\forall t \in(a,b),[x,x+h]\subset (a,b)$\\
  then:
  \[\phi(x+h) = \phi(x) + \phi^{(1)}(x)h + .. + \phi^{(n)}(x) \frac{h^n}{n!} + ..\]
  for some $\hat{x}\in(x,x+h)$\\
  \[
    f:\R^n \to \R,
    \nabla f(x) =
    \begin{bmatrix}
      \ppartial{f(x)}{x_1} \\
      ..\\
      \ppartial{f(x)}{x_n} \\
    \end{bmatrix},
    x =
    \begin{bmatrix}
      x_1\\ .. x_n
    \end{bmatrix}
  \]
  \[
    \nabla^2 f(x) =
    \begin{bmatrix}
      \frac{\p^2 f(x)}{\p x_1 \p x_1} & .. \\
      .. & \frac{\p^2 f(x)}{\p x_n \p x_n}
    \end{bmatrix}
  \]
  
  $f^{(2)}(x)$ is continuous $\implies$ mixed partial derivatives are commutative\\
  $f: \R^n \to \R$ is continuously differentiable $\implies \\ f(x+p)=f(x)+\nabla f(x+tp)^T p, \exists t\in(0,1)$
  \begin{align*}
    &\te{let} \phi(t) = f(x+tp)\\
    \phi'(t) &= \lim_{h\to 0} \frac{\phi(t+h)-\phi(t)}{h} = \lim_{h\to 0}\frac{f(x+(t+h)p)-f(x+tp)}{h}\\
    &\te{let }x\in\R^2\\
    \phi'(t) &= \lim_{h\to 0} \frac{f\bigg(
               \begin{bmatrix}
                 x_1+(t+h)p_1\\
                 x_2+(t+h)p_2
               \end{bmatrix}\bigg)
    -f\bigg(\begin{bmatrix}
      x_1+tp_1\\
      x_2+tp_2
    \end{bmatrix}\bigg)}{h}\\
    &=\lim_{h\to 0}
      \frac{f\bigg(\begin{bmatrix}
          x_1+(t+h)p_1\\
          x_2+(t+h)p_2
        \end{bmatrix}\bigg)
    -f\bigg(
    \begin{bmatrix}
      x_1+tp_1\\
      x_2+(t+h)p_2
    \end{bmatrix}\bigg)}{h}\\
    % &+\frac{f\bigg(
    % \begin{bmatrix}
    %   x_1+tp_1\\
    %   x_2+(t+h)p_2
    % \end{bmatrix}\bigg)
    % -f\bigg(\begin{bmatrix}
    %   x_1+tp_1\\
    %   x_2+tp_2
    % \end{bmatrix}
    % \bigg)}{h}\\
    &=\lim_{h\to 0}\frac{f\bigg(
      \begin{bmatrix}
        x_1+(t+h)p_1\\
        x_2+(t+h)p_2
      \end{bmatrix}
    \bigg)
    -f\bigg(
    \begin{bmatrix}
      x_1+tp_1\\
      x_2+(t+h)p_2
    \end{bmatrix}
    \bigg)}
    {h}\\
    &+\lim_{h\to 0}\frac{f\bigg(
      \begin{bmatrix}
        x_1+tp_1\\
        x_2+(t+h)p_2
      \end{bmatrix}
    \bigg)
    -f\bigg(
    \begin{bmatrix}
      x_1+tp_1\\
      x_2+tp_2
    \end{bmatrix}
    \bigg)}
    {h}
  \end{align*}

  \begin{align*}
    \phi'(t)
    &=\lim_{h\to 0}\frac{f\bigg(
      \begin{bmatrix}
        x_1+(t+h)p_1\\
        x_2+(t+h)p_2
      \end{bmatrix}
    \bigg)
    -f\bigg(
    \begin{bmatrix}
      x_1+tp_1\\
      x_2+(t+h)p_2
    \end{bmatrix}
    \bigg)}
    {hp_1}\frac{hp_1}{h}\\
    &+\lim_{h\to 0}\frac{f\bigg(
      \begin{bmatrix}
        x_1+tp_1\\
        x_2+(t+h)p_2
      \end{bmatrix}
    \bigg)
    -f\bigg(
    \begin{bmatrix}
      x_1+tp_1\\
      x_2+tp_2
    \end{bmatrix}
    \bigg)}
    {hp_2}\frac{hp_2}{h}\\
    &=\frac{\p f}{\p x_1}(x+tp)p_1 + \frac{\p f}{\p x_2}(x+tp)p_2\\
    f(x+tp)&=f(x) + \nabla f(x+tp)^Tp\\
    \phi(1)&=\phi(0)+\phi'(t)\\
    \phi''(t)&=p^T \nabla^2 f(x+tp)p\\
    \phi(1)&=\phi(0)+\phi'(0)+\frac{\phi''(t)}{2}, \text{ for some } t\in(0,1)\\
    f(x+p)&=f(x)+\nabla f(x)^T p + \frac{1}{2}p^T \nabla f(x+tp)p, f:\R^n\to \R\\
    \nabla f(x+p)&=\nabla f(x)+\int_0^1 \nabla^2 f(x+tp)p dt\\
    \phi(t)&=\nabla f(x+tp)_i \in \R\\
    \phi(1)&=\phi(0)+\int_o^1 \phi'(t)dt\\
    \nabla f(x+tp)_i&=\nabla f(x)_i+\int_0^1 \nabla^2 f(x+tp)_i^T p dt, \text{ i=ith row of Hessian matrix}\\
    \nabla f(x+tp)&=\nabla f(x) + \int_0^1 \nabla^2 f(x+tp)^T p dt\\
  \end{align*}

  note:\\
  $\nabla f(x+p)_i = \nabla f(x)_i + \nabla^2 f(x+t_i p)p$, for some $t_i\in(0,1)$ is true\\
  $\nabla f(x+p) = \nabla f(x) + \nabla^2 f(x+t p)p$, for some $t\in(0,1)$ is not true\\
  
  \pagebreak

  \section{1st order conditions}
  
  \begin{theorem}
    1st order necessary condition.\\
  if $x^*$ is a local minimizer of $f$ and $f$ is continuously differentiable in a beighbourhood of $x^*$ then $\nabla f(x^*)=0$\\
\end{theorem}
\begin{proof}
  By contradiction (assume hypothesis true, conclusion false)\\
  let $p= -\nabla f(x^*)$\\
  note $\nabla f(x^*)^T p = - \nabla f(x^*)^T \nabla f(x^*) < 0$\\
  consider $- \nabla f(x^*)^T \nabla f(x* + tp)<0$ for $t\in[0,\alpha], \alpha>0$\\
  $\phi(t)=\nabla f(x^*)^T \nabla f(x^*+tp), \phi: \R \to \R$\\
  $\phi(0)=\nabla f(x^*)^T \nabla f(x^*)> 0$ (using assumed false conclusion)\\
  $| \phi(o) -\phi(t)| < \epsilon = \frac{\phi(0)}{2}, |t| < \delta$\\
  by continuity of $\phi(t)$,\\
  $(\exists \alpha >0) - \nabla f(x^*)^T \nabla f(x^*+tp) <0, t\in[0,\alpha]$\\
  $f(x^*+\alpha p)=f(x^*)+\alpha \nabla f(x+tp)^T p$, for some $t\in[0,\alpha]$\\
  $\alpha \nabla f(x+tp)^T p<0$, then:\\
  $f(x^*+\alpha p) < f(x^*)$\\
  $x^*$ is a local minimizer\\
  there esits an $r>0$ s.t. for all $x\inB(x^*,r) f(x^*)\leqf(x)$\\
  note for $x=x^*+\alpha p: \norm{x-x^*}=|a|\norm{p}<r$\\
  we can adjust $t$ to all smaller values and $\alpha \nabla f(x+tp)^T p<0$ would still hold\\
  thus contradiction
\end{proof}
\end {document}
